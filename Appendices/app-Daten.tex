\chapter{Rohdaten} 

\section{Programmierfortschritte}

Die umfangreichen Programmierfortschritte-Daten stehen vollständig unter den folgenden URIs zur Verfügung:

\begin{description}
  \item[Workshop'11] \url{https://github.com/bkahlert/seqan-research/tree/master/raw/workshop11}
  \item[Workshop'12] \url{https://github.com/bkahlert/seqan-research/tree/master/raw/workshop12}
  \item[Workshop'13] \url{https://github.com/bkahlert/seqan-research/tree/master/raw/workshop13}
  \item[PMSB'12] \url{https://github.com/bkahlert/seqan-research/tree/master/raw/pmbs12}
  \item[PMSB'13] \url{https://github.com/bkahlert/seqan-research/tree/master/raw/pmsb13}
\end{description}

\section[Interviewnotizen ``ATLAS.ti'']{Notizen zum offenen Interviews zum Thema ``ATLAS.ti''}

Um eine valide Einschätzung von den Stärken und Schwächen der qualitativen Datenanalysesoftware \textit{ATLAS.ti} zu erhalten, habe ich jeweils ein offenes Interview mit meinen Arbeitsgruppen-Kollegen geführt. Die dabei angefertigten Notizen sind im Folgenden aufgeführt.

\subsection[Franz Zieris, 10.10.2014]{Notizen zum offenen Interview mit Franz Zieris am 10.10.2014}

\begin{description}
	\item[Aufbau] \hfill
\begin{itemize}
\itemsep1pt\parskip0pt\parsep0pt
\item Hermeneutische Einheiten (hermeneutic unit, HU)
\item z.B. Liste von Primärdokumenten, (z.B. P19)
\item Globale Zeitleiste
\item Gezoomte Zeitleiste
\item Code Manager
\item Phänomen / Quotations = Markiertes Datum / Bereich
\item Code = Konzept, wird an Phänomen gebunden
\end{itemize}

	\item[Pro] \hfill
\begin{itemize}
\itemsep1pt\parskip0pt\parsep0pt
\item Multimonitor
\item Eigenschaften sind Codes
\item Memos müssen nicht an Phänomene oder Codes gebunden sein
\item Phänomen kann mehrere Codes haben [in APIUA nicht möglich]
\item Memos haben einen Typ (z.B. Forschungsmemo)
\item Verschiedenste Medientypen möglich
\item Rich text editor für Memos (mit OLE Support)
\end{itemize}

	\item[Negativ] \hfill
\begin{itemize}
\itemsep1pt\parskip0pt\parsep0pt
\item Click auf Quotation $\rightarrow$ Zeitleiste springt an den Anfang des Phänomens (auch wenn man schon in der Mitte war) $\rightarrow$ verändert den Fokus
\item Zu viele Quotations führen durch die Anordnung nebeneinander dazu, dass sie nicht mehr auf den Bildschirm passen $\rightarrow$ horizontales Scrollen
\item Positionierung kann nicht verändert werden
\item Kein automatisches Clustering
    \begin{itemize}
    \itemsep1pt\parskip0pt\parsep0pt
    \item 1x Diagramm pro Code
    \item keine Gesamtübersicht
    \item Pfeilverbindungen können nicht ausgeblendet / beeinflusst werden
    \item Strategie: in Visio machen
    \end{itemize}
\item Schlechter Umgang mit großen Datenmengen [langsam?]
\item Keine Sortierung von Codes möglich
\item Keine Gruppierung von Codes möglich (bzw. nur 1-2 zusätzliche Levels mittels Familien und Superfamilien; schwierig zu handhaben: furchtbar komplizierte Syntax, IDs werden wiederverwendet [bei APIUA kommen IDs auf eine Sperrliste])
\item Kein Filtern von Quotations
\item Kein Überblick beim Open Coding
\item Fokus verrutscht leicht
\item Keine Referenzen zwischen Memos möglich [aber bei APIUA mittels Hyperlinks]
\item Code-Eigenschaften sind nur in einer separaten Übersicht zu sehen. Folge: schwer den Überblick zu behalten
\item Lösung: bessere Darstellung; Eigenschaft allgemein und Eigenschaftswert an Phänomen gebunden
\item Constant comparision schwierig: viele Ansicht sind exklusiv und nur Ausschnitte $\rightarrow$ zeitraubender Wechsel zwischen Ansichten (z.B. Daten- und Phänomenansicht, Doppelklick auf Phänomen schließt Phänomenansicht und öffen Datenansicht; beides soll offen bleiben)
\item Backup-Zeiten unbestimmt [bei APIUA nach jeder Aktion]
\item Suche sehr Umständlich wegen komplizierter Syntax
\item Quotationrahmen können nicht verschoben werden
\item Inkonsistente Icons (Speichern = Diskette und manchmal Haken)
\item Keine Codiersupport für mehr als einer Person
\end{itemize}
\end{description}



\subsection[Julia Schenk, 03.07.2014]{Notizen zum offenen Interview mit Julia Schenk am 03.07.2014}

\begin{description}
	\item[Positiv] \hfill
\begin{itemize}
\itemsep1pt\parskip0pt\parsep0pt
\item Frei gestaltbare UI (Videos abdocken und auf zweiten Bildschirm ziehen)
\item Drag'n'Drop-Support
\item Co-Occurrency tables
\end{itemize}

	\item[Negativ] \hfill
\begin{itemize}
\itemsep1pt\parskip0pt\parsep0pt
\item Co-Occurrency tables: Konfiguration kann nicht gespeichert werden
\item Zu großer Platzbedarf für die Anzeige von Quotations (führt zu horizontalem Scrollen)
\item Quotation nicht mit \texttt{Entfernen} löschbar
\item Der Versuch, ein Netzwerk zu überschreiben, wird als mit ``Unique name requested'' quittiert.
\end{itemize}
\end{description}






\section[Ausgefüllte Cognitive-Dimensions-Fragebögen, 15.09.2013]{Ausgefüllte Cognitive-Dimensions-Fragebögen Workshop'13 am 15.09.2013}

Die ausgefüllten Cognitive-Dimensions-Fragebögen können online unter \url{https://github.com/bkahlert/seqan-research/tree/master/raw/workshop13/workshop2013-data-20130926/cd} abgerufen werden.

%\inputminted{xml}{Data/2013-09-18T17-35-54.33460000+0200.xml}
%\inputminted{xml}{Data/2013-09-18T17-41-31.92928500+0200.xml}
%\inputminted{xml}{Data/2013-09-18T17-42-29.89830600+0200.xml}
%\inputminted{xml}{Data/2013-09-18T17-44-46.06047700+0200.xml}
%\inputminted{xml}{Data/2013-09-18T17-45-54.88891500+0200.xml}
%\inputminted{xml}{Data/2013-09-18T17-46-05.89001200+0200.xml}
%\inputminted{xml}{Data/2013-09-18T17-46-55.04209500+0200.xml}
%\inputminted{xml}{Data/2013-09-18T17-50-13.42530400+0200.xml}
%\inputminted{xml}{Data/2013-09-18T17-50-37.89996200+0200.xml}
%\inputminted{xml}{Data/2013-09-19T11-51-16.61646000+0200.xml}



\section[Gruppendiskussions-Transkript, 06.09.2014]{Transkript der Gruppendiskussion am 06.09.2014}
\label{app:gruppendiskussion}

Die Gruppendiskussion kann online abgerufen werden:

\begin{description}
  \item[Transkript] \url{https://rawgit.com/bkahlert/seqan-research/master/raw/workshop12/workshop2012-data-20120906/group-discussions/workshop'12\%20-\%20Interview\%20Gruppendiskussion\%20(2012-09-06T13-01-28+0200).html}
  \item[Audioaufzeichnung] \url{https://github.com/bkahlert/seqan-research/blob/master/raw/workshop12/workshop2012-data-20120906/group-discussions/workshop'12\%20-\%20Interview\%20Gruppendiskussion\%20(2012-09-06T13-01-28\%2B0200).mp4}
\end{description}