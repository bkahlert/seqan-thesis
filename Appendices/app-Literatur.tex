\glsresetall
\chapter{Literaturergänzungen}

\section[Heuristiken der Heuristischen Evaluation]{Heuristiken der Heuristischen Evaluation (Nielsen u. a. 1990)}
\label{app:heuristiken}

\textit{Übersetzung \citep[][S. 101]{Schweibenz:2003tu}}

\begin{enumerate}
    \item[H1] Sichtbarkeit des Systemstatus
        \\Das System soll die Benutzer ständig darüber informieren, was geschieht, und zwar durch eine angemessene Rückmeldung in einem vernünftigen zeitlichen Rahmen.
    \item[H2] Übereinstimmung zwischen dem System und der realen Welt
        \\Das System sollte die Sprache des Benutzers sprechen, und zwar nicht mir systemorientierter Terminologie, sondern mit Worten, Phrasen und Konzepten, die den Benutzern vertraut sind. Dabei soll die natürliche und logische Reihenfolge eingehalten werden.
    \item[H3] Benutzerkontrolle und -freiheit
        \\Benutzer wählen Systemfunktionen oft fälschlicherweise aus und benötigen einen 'Notausgang', um den unerwünschten Zustand wieder zu verlassen. Dazu dienen Undo- und Redo-Funktionen.
    \item[H4] Konsistenz und Standards
        \\Benutzer sollten sich nicht fragen müssen, ob verschiedene Begriffe oder Aktionen dasselbe bedeuten. Deshalb sind Konventionen einzuhalten.
    \item[H5] Fehlerverhütung
        \\Noch besser als gute Fehlermeldungen ist ein sorgfältiges Design, das Fehler verhütet.
    \item[H6] Wiedererkennen, statt sich erinnern
        \\Objekte, Optionen und Aktionen sollten sichtbar sein. Die Benutzer sollten sich nicht an Informationen aus einem früheren Teil des Dialogs mit dem System erinnern müssen. Instruktionen sollen sichtbar oder leicht auffindbar sein.
    \item[H7] Flexibilität und Effizienz der Benutzung
        \\Häufig auftretende Aktionen sollten vom Benutzer angepasst werden können, um Fortgeschrittenen eine schnellere Bedienung zu erlauben.
    \item[H8] Ästhetik und minimalistisches Design
        \\Dialoge sollten keine irrelevanten Informationen enthalten, da die Informationen um die Aufmerksamkeit des Benutzers konkurrieren.
    \item[H9] Hilfe beim Erkennen, Diagnostizieren und Beheben von Fehlern
        \\Fehlermeldungen sollten in natürlicher Sprache ausgedrückt werden (keine Fehlercodes), präzise das Problem beschreiben und konstruktiv eine Lösung vorschlagen.
    \item[H10] Hilfe und Dokumentation
        \\Jede Information der Hilfe oder Dokumentation sollte leicht zu finden sein, auf die Aufgabe abgestimmt sein und die konkreten Schritte zur Lösung auflisten. Außerdem sollte sie nicht zu lang sein. (Schweibenz \& Thissen, 2003, S. 101)
\end{enumerate}




\section[Cognitive Dimensions Fragebogen]{Sämtliche Fragen des ``Cognitive Dimensions Questionnaire Optimised for Users'' (Blackwell u. Green 2000)}
\label{app:cdf-questions}

\begin{itemize}
    \item Visibility and Juxtaposability
        \begin{itemize}
        \item How easy is it to see or find the various parts of the notation while it is being created or changed? Why?
        \item What kind of things are more difficult to see or find?
        \item If you need to compare or combine different parts, can you see them at the same time? If not, why not?
        \end{itemize}
    \item Viscosity
        \begin{itemize}
        \item When you need to make changes to previous work, how easy is it to make the change? Why?
        \item Are there particular changes that are more difficult or especially difficult to make? Which ones?
        \end{itemize}
    \item Diffuseness
        \begin{itemize}
        \item Does the notation a) let you say what you want reasonably briefly, or b) is it long-winded? Why?
        \item What sorts of things take more space to describe?
        \end{itemize}
    \item Hard Mental Operations
        \begin{itemize}
        \item What kind of things require the most mental effort with this notation?
        \item Do some things seem especially complex or difficult to work out in your head (e.g. when combining several things)? What are they?
        \end{itemize}
    \item Error Proneness
        \begin{itemize}
        \item Do some kinds of mistake seem particularly common or easy to make? Which ones?
        \item Do you often find yourself making small slips that irritate you or make you feel stupid? What are some examples?
        \end{itemize}
    \item Closeness of Mapping
        \begin{itemize}
        \item How closely related is the notation to the result that you are describing? Why? (Note that in a sub-device, the result may be part of another notation, rather than the end product).
        \item Which parts seem to be a particularly strange way of doing or describing something?
        \end{itemize}
    \item Role Expressiveness
        \begin{itemize}
        \item When reading the notation, is it easy to tell what each part is for in the overall scheme? Why?
        \item Are there some parts that are particularly difficult to interpret? Which ones?
        \item Are there parts that you really don’t know what they mean, but you put them in just because it’s always been that
way? What are they?
        \end{itemize}
    \item Hidden Dependencies
        \begin{itemize}
        \item If the structure of the product means some parts are closely related to other parts, and changes to one may affect the other, are those dependencies visible? What kind of dependencies are hidden?
        \item In what ways can it get worse when you are creating a particularly large description?
        \item Do these dependencies stay the same, or are there some actions that cause them to get frozen? If so, what are they?
        \end{itemize}
    \item Progressive Evaluation
        \begin{itemize}
        \item How easy is it to stop in the middle of creating some notation, and check your work so far? Can you do this any time you like? If not, why not?
        \item Can you find out how much progress you have made, or check what stage in your work you are up to? If not, why not?
        \item Can you try out partially-completed versions of the product? If not, why not?
        \end{itemize}
    \item Provisionality
        \begin{itemize}
        \item Is it possible to sketch things out when you are playing around with ideas, or when you aren’t sure which way to proceed? What features of the notation help you to do this?
        \item What sort of things can you do when you don’t want to be too precise about the exact result you are trying to get?
        \end{itemize}
    \item Premature Commitment
        \begin{itemize}
        \item When you are working with the notation, can you go about the job in any order you like, or does the system force you to think ahead and make certain decisions first?
        \item If so, what decisions do you need to make in advance? What sort of problems can this cause in your work?
        \end{itemize}
    \item Consistency
        \begin{itemize}
        \item Where there are different parts of the notation that mean similar things, is the similarity clear from the way they appear? Please give examples.
        \item Are there places where some things ought to be similar, but the notation makes them different? What are they?
        \end{itemize}
    \item Secondary Notation
        \begin{itemize}
        \item Is it possible to make notes to yourself, or express information that is not really recognised as part of the notation?
        \item If it was printed on a piece of paper that you could annotate or scribble on, what would you write or draw?
        \item Do you ever add extra marks (or colours or format choices) to clarify, emphasise or repeat what is there already? [If
yes: does this constitute a helper device? If so, please fill in one of the section 5 sheets describing it]
        \end{itemize}
    \item Abstraction Management
        \begin{itemize}
        \item Does the system give you any way of defining new facilities or terms within the notation, so that you can extend it to describe new things or to express your ideas more clearly or succinctly? What are they?
        \item Does the system insist that you start by defining new terms before you can do anything else? What sort of things?
        \item If you wrote here, you have a redefinition device: please fill in one of the section 5 sheets describing it.
        \end{itemize}
    \end{itemize}